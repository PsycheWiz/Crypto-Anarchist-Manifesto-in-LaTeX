\documentclass[12pt]{article}
\usepackage[left=2cm, right=2cm, top=1cm, bottom=1cm]{geometry}
\usepackage{ragged2e}
\usepackage{charter}

\title{\textup{\Huge{\underline{The Crypto Anarchist Manifesto}}}}
\author{\textbf{\LARGE{Timothy.C.May}}}
\date{\textup{\Large{22 Nov 92 12:11:24 PST}}}

\begin{document}
\maketitle
\thispagestyle{empty}

\begin{flushleft}
\begin{justify} 
\textsl{A specter is haunting the modern world, the specter of crypto anarchy.}
Computer technology is on the verge of providing the ability for individuals
and groups to communicate and interact with each other in a totally anonymous
manner. Two persons may exchange messages, conduct business, and negotiate
electronic contracts without ever knowing the True Name, or legal identity, of
the other. Interactions over networks will be untraceable, via extensive
re-routing of encrypted packets and tamper-proof boxes which implement
cryptographic protocols with nearly perfect assurance against any tampering.
Reputations will be of central importance, far more important in dealings than
even the credit ratings of today. These developments will alter completely the
nature of government regulation, the ability to tax and control economic
interactions, the ability to keep information secret, and will even alter the
nature of trust and reputation.
\\
The technology for this revolution--and it surely will be both a social and
economic revolution--has existed in theory for the past decade. The methods are
based upon public-key encryption, zero-knowledge interactive proof systems, and
various software protocols for interaction, authentication, and verification.
The focus has until now been on academic conferences in Europe and the U.S.,
conferences monitored closely by the National Security Agency. But only
recently have computer networks and personal computers attained sufficient
speed to make the ideas practically realizable. And the next ten years will
bring enough additional speed to make the ideas economically feasible and
essentially unstoppable. High-speed networks, ISDN, tamper-proof boxes, smart
cards, satellites, Ku-band transmitters, multi-MIPS personal computers, and
encryption chips now under development will be some of the enabling
technologies.
\\
The State will of course try to slow or halt the spread of this technology,
citing national security concerns, use of the technology by drug dealers and
tax evaders, and fears of societal disintegration. Many of these concerns will
be valid; crypto anarchy will allow national secrets to be trade freely and
will allow illicit and stolen materials to be traded. An anonymous computerized
market will even make possible abhorrent markets for assassinations and
extortion. Various criminal and foreign elements will be active users of
CryptoNet. But this will not halt the spread of crypto anarchy.
Just as the technology of printing altered and reduced the power of medieval
guilds and the social power structure, so too will cryptologic methods
fundamentally alter the nature of corporations and of government interference
in economic transactions. Combined with emerging information markets, crypto
anarchy will create a liquid market for any and all material which can be put
into words and pictures. And just as a seemingly minor invention like barbed
wire made possible the fencing-off of vast ranches and farms, thus altering
forever the concepts of land and property rights in the frontier West, so too
will the seemingly minor discovery out of an arcane branch of mathematics come
to be the wire clippers which dismantle the barbed wire around intellectual
property. \textsl{Arise, you have nothing to lose but your barbed wire fences!}
\end{justify}
\end{flushleft}
\end{document}
